\chapter{Conclusion}
brief of conclusion
\maxdeadcycles=200
\extrafloats{100}

\section{Cokro Edi Prawiro / 1164069}

\subsection{Teori}
\begin{enumerate}

\item Jelaskan Kenapa Kata-Kata harus dilakukan vektorisasi lengkapi dengan ilustrasi gambar.\par
Kata kata harus dilakukan vektorisasi dikarenakan atau bertujuan utk mengukur nilai kemunculan suatau kata yang serupa dari sebuah kalimat sehingga kata-kata tersebut dapat di prediksi kemunculanya. atau juga di buatnya vektorisasi data digunakan untuk memprediksi bobot dari suatu kata misalkan ayam dan kucing sama-sama hewan maka akan dibuat prediksi apakah kata tersebut akan muncul pada kalimat yang kira-kira memiliki bobot yang sama. untuk dapat jelasnya dapat melihat ilustrasi pada gambar \ref{c71}.

\begin{figure}[!htbp]
      \centering{\includegraphics[width=0.5\textwidth]
      {figures/cokro/c71}}
      \caption{Ilustrasi Vektorisasi Kata-Kata}
      \label{c71}
      \end{figure}

\item Jelaskan Mengapa dimensi dari vektor dataset google bisa mencapai 300 lengakapi dengan ilustrasi gambar. \par
Dimensi dataset dari google bisa mencapai 300 karena dimensi dari vektor tersebut digunakan untuk membandingkan bobot dari setiap kata, misalkan terdapat kata dog dan cat pada dataset google tersebut setiap kata tersebut di buat dimensi vektor 300 untuk kata dog dan 300 dimensi vektor juga untuk kata cat kemudian kata tersebt di bandingkan bobot kesamaan katanya maka akan muncul akurasi sekitar 70 persen kesamaan bobot dikarenakan kata dog dan cat sama sama di gunakan untuk hewan priharaan. untuk lebih jelasnya dapat dilihat pada gambar \ref{c72}.

\begin{figure}[!htbp]
      \centering{\includegraphics[width=0.5\textwidth]
      {figures/cokro/c72}}
      \caption{Ilustrasi Kenapa dimensi vektor pada datasets google haris 300}
      \label{c72}
      \end{figure}

\item Jelaskan Konsep vektorisasi untuk kata . dilengkapi dengan ilustrasi atau gambar. \par
Vektorisasi untuk kata untuk mengetahui kata tengah dari suatau kalimat atau kata utama atau objek utama pada suatau kalimat contoh ( Jangan lupa subscribe channel saya ya sekian treimakasih ) kata tengah tersebut merupakan channel yang memiliki bobot sebagai kata tengah dari suatu kalimat atau bobot sebagai objek dari suatu kalimat. hal ini sangat berkaitan dengan dimensi vektor pada dataset google yang 300 tadi karena untuk mendapatkan nilai atau bobot dari kata tengah tersebut di dapatkan dari proses dimensiasi dari kata tersebut. untuk lebih jelasnya dapat dilihat pada gambar \ref{c73} berikut :

\begin{figure}[!htbp]
      \centering{\includegraphics[width=0.5\textwidth]
      {figures/cokro/c73}}
      \caption{Ilustrasi konsep vektorisasi untuk kata}
      \label{c73}
      \end{figure}

\item Jelaskan Konsep vektorisasi untuk dokumen. dilengkapi dengan ilustrasi atau gambar. \par
Vektorisasi untuk dokumen hampir sama seperti vektorisasi untuk kata hanya saja pemilihan kata utama atau kata tengah terdapat pada satu dokumen jadi mesin akan membuat dimensi vektor 300 untuk dokumen dan nanti kata tengahnya akan di sandingkan pada dokumen yany terdapat pada dokumen tersebut contoh dapat dilihat pada gambar \ref{c74} berikut : 

\begin{figure}[!htbp]
      \centering{\includegraphics[width=0.5\textwidth]
      {figures/cokro/c74}}
      \caption{Ilustrasi konsep vektorisasi untuk dokumen}
      \label{c74}
      \end{figure}

\item Jelaskan apa mean dan standar deviasi, lengkapi dengan iludtrasi atau gambar. \par
mean merupakan petunjuk terhadap kata-kata yang di olah jika kata kata itu akurasinya tinggi berarti kata tersebut sering muncul begitu juga sebaliknya untuk lebih jelasnya dapat dilihat pada gambar \ref{c75} sedangkan setandar defiation merupakan standar untuk menimbang kesalahan. sehingga kesalahan tersebut di anggap wajar misarkan kita memperkirakan kedalaman dari dataset merupakan 2 atau 3 tapi pada kenyataanya merupakan 5 itu merupakan kesalahan tapi masih bisa dianggap wajar karna masih mendekati perkiraan awal.

\begin{figure}[!htbp]
      \centering{\includegraphics[width=0.5\textwidth]
      {figures/cokro/c75}}
      \caption{Ilustrasi Penggunaan Mean}
      \label{c75}
      \end{figure}

\item Jelaskan Apa itu Skip-Gram sertakan contoh ilustrasi. \par
Skip-Gram adalah kebalikan dari konsep vektorisasi untuk kata dimana kata tengah menjadi acuan terhadap kata kata pelengkap dalam suatu kalimat untuk lebih jelasnya dapat di lihat pada gambar \ref{c76} berikut :

\begin{figure}[!htbp]
      \centering{\includegraphics[width=0.5\textwidth]
      {figures/cokro/c76}}
      \caption{Ilustrasi Skip-Gram}
      \label{c76}
      \end{figure}

\end{enumerate}
\subsection{Praktikum}
\begin{enumerate}
\item mencoba dataset google dan penjelasan vektor dari kata love, faith, fall, sick, clear, shine, bag, car, wash, motor, dan cycle.\par

\begin{itemize}
\item berikut merupakan code import gensim digunakan untuk membuat data model atau rangcangan data yang akan di buat. selanjutnya dibuat variabel gmodel yang berisi data vektor negativ. selanjutnya data tersebut di load agar data tersebut dapat di tampilkan dan di olah. code lengkapnya dapat dilihat pada gambar \ref{c89}.

\begin{figure}[!htbp]
      \centering{\includegraphics[width=0.5\textwidth]
      {figures/cokro/c89}}
      \caption{Import Gensim dan membuat model gmodel}
      \label{c89}
      \end{figure}

\item berikut merupakan hasil lpengolahan kata clear pada data google yang di load tadi. Sehingga memunculkan hasil vektor 300 dimensi untuk kata tersebut. untuk jelasnya dapat di lihat pada gambar \ref{c77}.
\begin{figure}[!htbp]
      \centering{\includegraphics[width=0.5\textwidth]
      {figures/cokro/c77}}
      \caption{Hasil Matrix Clear}
      \label{c77}
      \end{figure}

\item berikut merupakan hasil lpengolahan kata Shine pada data google yang di load. Sehingga memunculkan hasil vektor 300 dimensi untuk kata tersebut. untuk jelasnya dapat di lihat pada gambar \ref{c78}.

\begin{figure}[!htbp]
      \centering{\includegraphics[width=0.5\textwidth]
      {figures/cokro/c78}}
      \caption{Hasil Matrix Shine}
      \label{c78}
      \end{figure}

\item berikut merupakan hasil lpengolahan kata bag pada data google yang di load. Sehingga memunculkan hasil vektor 300 dimensi untuk kata tersebut. untuk jelasnya dapat di lihat pada gambar \ref{c79}.

\begin{figure}[!htbp]
      \centering{\includegraphics[width=0.5\textwidth]
      {figures/cokro/c79}}
      \caption{Hasil Matrix bag}
      \label{c79}
      \end{figure}

\item berikut merupakan hasil lpengolahan kata Car pada data google yang di load. Sehingga memunculkan hasil vektor 300 dimensi untuk kata tersebut. untuk jelasnya dapat di lihat pada gambar \ref{c80}.

\begin{figure}[!htbp]
      \centering{\includegraphics[width=0.5\textwidth]
      {figures/cokro/c80}}
      \caption{Hasil Matrix Car}
      \label{c80}
      \end{figure}


\item berikut merupakan hasil lpengolahan kata Wash pada data google yang di load. Sehingga memunculkan hasil vektor 300 dimensi untuk kata tersebut. untuk jelasnya dapat di lihat pada gambar \ref{c81}.

\begin{figure}[!htbp]
      \centering{\includegraphics[width=0.5\textwidth]
      {figures/cokro/c81}}
      \caption{Hasil Matrix Wash}
      \label{c81}
      \end{figure}

\item berikut merupakan hasil lpengolahan kata Motor pada data google yang di load. Sehingga memunculkan hasil vektor 300 dimensi untuk kata tersebut. untuk jelasnya dapat di lihat pada gambar \ref{c82}.

\begin{figure}[!htbp]
      \centering{\includegraphics[width=0.5\textwidth]
      {figures/cokro/c82}}
      \caption{Hasil Matrix Motor}
      \label{c82}
      \end{figure}

\item berikut merupakan hasil lpengolahan kata Cycle pada data google yang di load. Sehingga memunculkan hasil vektor 300 dimensi untuk kata tersebut. untuk jelasnya dapat di lihat pada gambar \ref{c83}.

\begin{figure}[!htbp]
      \centering{\includegraphics[width=0.5\textwidth]
      {figures/cokro/c83}}
      \caption{Hasil Matrix Cycle}
      \label{c83}
      \end{figure}

\item berikut merupakan hasil lpengolahan kata love pada data google yang di load. Sehingga memunculkan hasil vektor 300 dimensi untuk kata tersebut. untuk jelasnya dapat di lihat pada gambar \ref{c84}.

\begin{figure}[!htbp]
      \centering{\includegraphics[width=0.5\textwidth]
      {figures/cokro/c84}}
      \caption{Hasil Matrix love}
      \label{c84}
      \end{figure}

\item berikut merupakan hasil lpengolahan kata faith pada data google yang di load. Sehingga memunculkan hasil vektor 300 dimensi untuk kata tersebut. untuk jelasnya dapat di lihat pada gambar \ref{c85}.

\begin{figure}[!htbp]
      \centering{\includegraphics[width=0.5\textwidth]
      {figures/cokro/c85}}
      \caption{Hasil Matrix faith}
      \label{c85}
      \end{figure}

\item berikut merupakan hasil lpengolahan kata Fall pada data google yang di load. Sehingga memunculkan hasil vektor 300 dimensi untuk kata tersebut. untuk jelasnya dapat di lihat pada gambar \ref{c86}.

\begin{figure}[!htbp]
      \centering{\includegraphics[width=0.5\textwidth]
      {figures/cokro/c86}}
      \caption{Hasil Matrix Fall}
      \label{c86}
      \end{figure}

\item berikut merupakan hasil lpengolahan kata Sick pada data google yang di load. Sehingga memunculkan hasil vektor 300 dimensi untuk kata tersebut. untuk jelasnya dapat di lihat pada gambar \ref{c87}.

\begin{figure}[!htbp]
      \centering{\includegraphics[width=0.5\textwidth]
      {figures/cokro/c87}}
      \caption{Hasil Matrix Sick}
      \label{c87}
      \end{figure}

\item berikut merupakan hasil dari similaritas kata kata yang di olah menjadi matrix tadi adapun persentase untuk perbandingan setiap katanya yaitu 9 persen untuk kata wash dan clear 7 persen untuk kata bag dan love 48 persen untuk kata motor dan car 12 persen untuk kata sick dan faith dan terakhir yaitu 6 persen untuk kata cycle dan shine. untuk jelasnya dapat di lihat pada gambar \ref{c88}.

\begin{figure}[!htbp]
      \centering{\includegraphics[width=0.5\textwidth]
      {figures/cokro/c88}}
      \caption{hasil dari lima similaritas}
      \label{c88}
      \end{figure}

\end{itemize}

\item pada code berikut merupakan hasil dari running code untu ekstrak word dimana pada baris ke tiga dimasukan perintah untuk menghapus tag html yang terdapat dalam file tesebut selanjutnya pada baris ke 4 yaitu perintah untuk menghilangkan tanda kutip satu selanjutnya pada baris ke 5 yaitu perintah untuk menghapus tanda baca pada file tersebut dan yang terakhir yaitu perintah untuk menghapus dable sepasi atau sepasi berurutan. setelah itu dibuat clas bari dari random yang bertujuan untuk mengkocok data yang ada pada file tersebut kemudian class permute sentence tersebut akan di gunkan untuk mengolah data selanjutnya. untuk lebih jelasnya dapat dilihat pada gambar \ref{c90}.

\begin{figure}[!htbp]
      \centering{\includegraphics[width=0.5\textwidth]
      {figures/cokro/c90}}
      \caption{hasil dari ekstrak\_words dan permute Sentence}
      \label{c90}
      \end{figure}

\end{enumerate}


\section{Fathi Rabbani / 1164074}
\subsection{Teori}
\begin{enumerate}
\item Why words need to be Vectorizer
\subitem karena kata - kata yang digunakan untuk memproses data agar dapat menjadi bagian dati kumpulan data atatu atribut yang dapat dibaca oleh sistem machine learning karena sistem tersebut tidak dapat memproses data text secara langsung dan harus di convert terlebih dahulu kedalam bilangan.untuk ilustrasinya dapat dilihat pada gambar \ref{fig1}
\begin{figure}[!htbp]
	\centering
	\includegraphics[width=0.5\textwidth]{figures/fathi/chapter5/hari1/1}
	\caption{Ilustrasi Vektorisasi Kata}
	\label{fig1}
\end{figure}

\item Why Dimension of Google dataset can reach 300
\subitem Dimensi dataset dari google bisa mencapai 300 karena dimensi dari vektor tersebut digunakan untuk membandingkan bobot dari setiap data kata yang diproses. ilustrasi dapat dilihat pada gambar \ref{fig2}
\begin{figure}[!htbp]
	\centering
	\includegraphics[width=0.5\textwidth]{figures/fathi/chapter5/hari1/2}
	\caption{Ilustrasi Google dataset}
	\label{fig2}
\end{figure}

\item Concept of Vectorizer on words
\subitem pada vektorisasi dengan menggunakan Word2Vec memiliki kelebihan yang dapat dibedakan dengan pengguaan bag of words yang biasanya.  pada bag of word pemrosesan data tidak dapat menganalisa data yang memiliki makna sama namun penulisannya berbeda, namun pada penggunaan Word2Vec proses tersebut dapat berjalan dengan lebih mudah contohnya adalah penulisan kata please dengan plz. untuk ilustrasi datanya bisa dilihat dalam gambar \ref{fig3}
\begin{figure}[!htbp]
	\centering
	\includegraphics[width=0.5\textwidth]{figures/fathi/chapter5/hari1/3}
	\caption{Ilustrasi Concept of Vectorizer on Words}
	\label{fig3}
\end{figure}

\item Concept of Vectorizer on documents
\subitem vektorisasi pada Doct2Vec dimana data yang terdapat pada file document tersebut diolah dengan melakukan pemrosesan yang mengutamakan nilai data filenamenya atau atribut utama dimana nilai data inputnya tidak terlalu diproses. ilustrasinya dapat dilihat pada gambar \ref{fig4}
\begin{figure}[!htbp]
	\centering
	\includegraphics[width=0.5\textwidth]{figures/fathi/chapter5/hari1/4}
	\caption{Ilustrasi Concept of Vectorizer on Document}
	\label{fig4}
\end{figure}

\item What is mean and deviation standart
\subitem Mean adalah nilai rata-rata dari beberapa buah data. Nilai mean dapat ditentukan dengan membagi jumlah data dengan banyaknya data.

\subitem Standar deviasi adalah nilai statistik yang digunakan untuk menentukan bagaimana sebaran data dalam sampel, dan seberapa dekat titik data individu ke mean – atau rata-rata – nilai sampel.

untuk ilustrasi data mean dan deviation standart bisa dilihat pada gambar \ref{fig5}
\begin{figure}[!htbp]
	\centering
	\includegraphics[width=0.5\textwidth]{figures/fathi/chapter5/hari1/5}
	\caption{Ilustrasi Mean and Deviation Standart}
	\label{fig5}
\end{figure}

\item What is skip-gram
\subitem Skip-gram merupakan teknik yang digunakan di area speech processing, dimana n-gram yang dibentuk kemudian ditambahkan juga dengan tindakan “skip” pada token-tokennya. contohnya terdapat pada gambar \ref{fig6}
\begin{figure}[!htbp]
	\centering
	\includegraphics[width=0.5\textwidth]{figures/fathi/chapter5/hari1/6}
	\caption{Ilustrasi  skip-gram}
	\label{fig6}
\end{figure}
\end{enumerate}

\subsection{Praktikum}
\begin{enumerate}
\item Try datasets GoogleNews-vectors
\begin{itemize}
\item berikut adalah hasil dari code yang digunakan untuk memanggil data library GENSIM dengan menggunakan perintah import, lalu dari library tersebut diambillah data yang akan digunakan untuk memproses data dari GoogleNews-vector. ilustrasi dapat dilihat pada gambar \ref{fig7}
\begin{figure}[!htbp]
	\centering
	\includegraphics[width=0.8\textwidth]{figures/fathi/chapter5/hari2/1}
	\caption{Ilustrasi  import gensim dan olah data GoogleNews-vector}
	\label{fig7}
\end{figure}

\item lalu pada penggunaan code berikut ini akan mengolah data LOVE yang terdapat pada file GoogleNews-vector, hasil dari pemrosesannya dapat dilihat pada gambar \ref{fig8}
\begin{figure}[!htbp]
	\centering
	\includegraphics[width=0.5\textwidth]{figures/fathi/chapter5/hari2/2}
	\caption{Ilustrasi hasil olah data LOVE pada GoogleNews-vector}
	\label{fig8}
\end{figure}

\item  lalu pada penggunaan code berikut ini akan mengolah data FAITH yang terdapat pada file GoogleNews-vector, hasil dari pemrosesannya dapat dilihat pada gambar \ref{fig8}
\begin{figure}[!htbp]
	\centering
	\includegraphics[width=0.5\textwidth]{figures/fathi/chapter5/hari2/3}
	\caption{Ilustrasi hasil olah data FAITH pada GoogleNews-vector}
	\label{fig8}
\end{figure}

\item  lalu pada penggunaan code berikut ini akan mengolah data FALL yang terdapat pada file GoogleNews-vector, hasil dari pemrosesannya dapat dilihat pada gambar \ref{fig9}
\begin{figure}[!htbp]
	\centering
	\includegraphics[width=0.5\textwidth]{figures/fathi/chapter5/hari2/4}
	\caption{Ilustrasi hasil olah data FALL pada GoogleNews-vector}
	\label{fig9}
\end{figure}

\item  lalu pada penggunaan code berikut ini akan mengolah data SICK yang terdapat pada file GoogleNews-vector, hasil dari pemrosesannya dapat dilihat pada gambar \ref{fig10}
\begin{figure}[!htbp]
	\centering
	\includegraphics[width=0.5\textwidth]{figures/fathi/chapter5/hari2/5}
	\caption{Ilustrasi hasil olah data SICK pada GoogleNews-vector}
	\label{fig10}
\end{figure}

\item  lalu pada penggunaan code berikut ini akan mengolah data CLEAR yang terdapat pada file GoogleNews-vector, hasil dari pemrosesannya dapat dilihat pada gambar \ref{fig11}
\begin{figure}[!htbp]
	\centering
	\includegraphics[width=0.5\textwidth]{figures/fathi/chapter5/hari2/6}
	\caption{Ilustrasi hasil olah data CLEAR pada GoogleNews-vector}
	\label{fig11}
\end{figure}

\item  lalu pada penggunaan code berikut ini akan mengolah data SHINE yang terdapat pada file GoogleNews-vector, hasil dari pemrosesannya dapat dilihat pada gambar \ref{fig12}
\begin{figure}[!htbp]
	\centering
	\includegraphics[width=0.5\textwidth]{figures/fathi/chapter5/hari2/7}
	\caption{Ilustrasi hasil olah data SHINE pada GoogleNews-vector}
	\label{fig12}
\end{figure}

\item  lalu pada penggunaan code berikut ini akan mengolah data BAG yang terdapat pada file GoogleNews-vector, hasil dari pemrosesannya dapat dilihat pada gambar \ref{fig13}
\begin{figure}[!htbp]
	\centering
	\includegraphics[width=0.5\textwidth]{figures/fathi/chapter5/hari2/8}
	\caption{Ilustrasi hasil olah data BAG pada GoogleNews-vector}
	\label{fig13}
\end{figure}

\item  lalu pada penggunaan code berikut ini akan mengolah data CAR yang terdapat pada file GoogleNews-vector, hasil dari pemrosesannya dapat dilihat pada gambar \ref{fig14}
\begin{figure}[!htbp]
	\centering
	\includegraphics[width=0.5\textwidth]{figures/fathi/chapter5/hari2/9}
	\caption{Ilustrasi hasil olah data CAR pada GoogleNews-vector}
	\label{fig14}
\end{figure}

\item  lalu pada penggunaan code berikut ini akan mengolah data WASH yang terdapat pada file GoogleNews-vector, hasil dari pemrosesannya dapat dilihat pada gambar \ref{fig15}
\begin{figure}[!htbp]
	\centering
	\includegraphics[width=0.5\textwidth]{figures/fathi/chapter5/hari2/10}
	\caption{Ilustrasi hasil olah data WASH pada GoogleNews-vector}
	\label{fig15}
\end{figure}

\item  lalu pada penggunaan code berikut ini akan mengolah data MOTOR yang terdapat pada file GoogleNews-vector, hasil dari pemrosesannya dapat dilihat pada gambar \ref{fig16}
\begin{figure}[!htbp]
	\centering
	\includegraphics[width=0.5\textwidth]{figures/fathi/chapter5/hari2/11}
	\caption{Ilustrasi hasil olah data MOTOR pada GoogleNews-vector}
	\label{fig16}
\end{figure}

\item  lalu pada penggunaan code berikut ini akan mengolah data CYCLE yang terdapat pada file GoogleNews-vector, hasil dari pemrosesannya dapat dilihat pada gambar \ref{fig17}
\begin{figure}[!htbp]
	\centering
	\includegraphics[width=0.5\textwidth]{figures/fathi/chapter5/hari2/12}
	\caption{Ilustrasi hasil olah data CYCLE pada GoogleNews-vector}
	\label{fig17}
\end{figure}

\item  dan pada hasil code berikut ini adalah hasil dari proses penggunaan perintah code similarity yang akan menghitung nilai value data yang dibandingkan dengan masing - masing kata seperti pada hasil dari perbandingan kata LOVE disandingkan dengan FAITH menghasilkan nilai 37 persen, sedangkan kata WASH dan SHINE menghasilkan nilai 27 persen dan kata CAR yang disandingkan dengan kata MOTOR menghasilkan 48 persen, dimana kita dapat menyimpulkan bahwa semakin data kata tersebut memiliki tingkat kesamaan yang tinggi maka nilai hasil yang ditampilkanpun akan semakin tinggi. ilustrasi bisa dilihat pada gambar \ref{fig18}
\begin{figure}[!htbp]
	\centering
	\includegraphics[width=0.5\textwidth]{figures/fathi/chapter5/hari2/13}
	\caption{Ilustrasi hasil olah data  pada GoogleNews-vector menggunakan SIMILARITY}
	\label{fig18}
\end{figure}
\end{itemize}

\item extract\_words dan PermutedSentences
\subitem pada penjelasan berikut ini akan menyangkut pembersihan data yang akan digunakan untuk diproses, dimana data akan di EXTRACT dari setiap katanya agar terbebas dari data TAG HTML, APOSTROPHES, TANDA BACA, dan SPASI yang berlebih. dengan menggunakan perintah code STRIP dan SPLIT. lalu penggunaan library random yang akan dibuat untuk melakukan KOCLOK data dengan acuan datanya adalah data yang terdapat pada variable KATA. untuk ilustrasi hasil dari codenya dapat dilihat pada gambar \ref{fig19}
\begin{figure}[!htbp]
	\centering
	\includegraphics[width=0.5\textwidth]{figures/fathi/chapter5/hari2/14}
	\caption{Ilustrasi hasil olah data  pada GoogleNews-vector menggunakan extract\_words dan PermuteSentences}
	\label{fig19}
\end{figure}
\end{enumerate}